\documentclass{article}
\usepackage[utf8]{inputenc}
\usepackage{mathtools,amssymb}
\usepackage{lipsum}
\usepackage{layout}
\usepackage{geometry}
\newcommand*{\vertbar}[1][10ex]{\rule[-1ex]{0.5pt}{#1}}
\newcommand*{\horzbar}{\rule[.5ex]{2.5ex}{0.5pt}}
\newcommand{\norm}[2]{\left\lVert#1\right\rVert_#2}
\newcommand{\abs}[1]{\lvert#1\rvert}

\title{STATS200 class notes}
\author{Erich Trieschman}
\date{2021 Fall quarter}

\newcommand{\userMarginInMm}{20}
\geometry{
%  legal,
 left=\userMarginInMm mm,
 right=\userMarginInMm mm,
 top=\userMarginInMm mm,
 bottom=20mm,
 footskip=5mm}


\begin{document}
\maketitle

% YOU ARE ON PAGE 49 OF THE NOTES!!!!!!!!!!!!!!!!!!!!!!!!!!!!!!!!!!!!!!!!!!!!
% COMBINATORICS AND PROBABILITY REVIEW
\section{Review: Combinatorics and probability}
%Logs, derivatives, integration, taylor expansions
\subsection{Calculus cheat sheet}
\subsubsection{Logs}
\begin{itemize}
	\item $log_b(M * N) = Lob_bM + log_bN$
	\item $log_b(\frac{M}{N}) = log_bM - log_bN$
	\item $log_b(M^k) = klog_bM$
	\item $e^ne^m = e^{n+m}$
\end{itemize}
\subsubsection{Derivatives}
\begin{itemize}
	\item $(x^n)' = nx^{n-1}$
	\item $(e^x)' = e^x$
	\item $(e^{u(x)})' = u'(x)e^x$
	\item $(log_e(x))' = (lnx)' = \frac{1}{x}$
	\item $(f(g(x)))' = f'(g(x))g'(x)$
\end{itemize}
\subsubsection{Integrals}
\begin{itemize}
	\item TODO: Integration by parts
\end{itemize}
\subsubsection{Infinite series and sums}
\begin{itemize}
	\item $e^x = 1 + x + \frac{x^2}{2!} + \frac{x^3}{3!} + \dots = \sum_{n=0}^\infty \frac{x^n}{n!}$
	\item $\frac{1}{1-x} = 1 + x + x^2 + x^3 + \dots = \sum_{n=0}^\infty a^x$ for $\abs{x} < 1$
	\item $ln(1 + x) = 1 - x + \frac{x^2}{2} - \frac{x^3}{3} + \dots = \sum_{n=0}^\infty (-1)^n\frac{x^n}{n}$
	\item $(1 + \frac{a}{n})^n \longrightarrow e^a$
\end{itemize}

% EVENTS AND SETS
\subsection{Events and sets}
Set operations follow commutative, associative, and distributive laws:
\begin{itemize}
    \item Commutative: $E \cup F = F \cup E$ and $E\cap F = F \cap E $ (also written $EF = FE$)
    \item Associative: $(E\cup F)\cup G = E \cup (f\cup G)$ and $(E\cap F)\cap G = E\cap(F\cap G)$
    \item Distributive: $(E\cup F)\cap G = (E\cap G) \cup (F \cap G) = E\cap G \cup F \cap G$ and $E\cap F\cup G = (E\cup G) \cap (F \cup G) = E\cup G \cap F \cup G$
\end{itemize}
\textbf{DeMorgan's Laws} relate the complement of a union to the intersection of complements:
\begin{itemize}
    \item $(\cup_{i=1}^n E_i)^c = \cap_{i=1}^nE_i^c$
    \item $(\cap_{i=1}^n E_i)^c = \cup_{i=1}^nE_i^c$
\end{itemize}

% PROBABILITY
\subsection{Probability}
A \textbf{probability space} is defined by a triple of objects $(S, \mathcal{E}, P)$:
\begin{itemize}
    \item $S:$ Sample space
    \item $\mathcal{E}:$ Set of possible events within the sample space. Set of events are assumed to be $\theta$-field (below)
    \item $P:$ Probability for each event
\end{itemize}
A \textbf{$\theta$-field} is a collection of subsets $\mathcal{E} \subset S$ that satisfy
\begin{itemize}
    \item $0 \in \mathcal{E}$
    \item $E \in \mathcal{E} \Rightarrow E^C \in \mathcal{E}$
    \item $E_i \in \mathcal{E}$ for {$1, 2, \dots$} $\Rightarrow \cup_{i=1}^\infty E_i \in \mathcal{E}$
\end{itemize}
\textbf{Basic probability properties}
\begin{itemize}
	\item $P(A^C)=1 - P(A)$
	\item $P(0)=0$
	\item $A\subset B \longrightarrow P(A) \leq P(B)$
	\item $P(A \cup B)=P(A)+P(B) - P(A \cap B)$ 
\end{itemize}
The \textbf{law of total probability} relates marginal probabilities to conditional probabilities. For a partition, {$E_1, E_2, \dots$} of set, $S$, where a partition implies i) $E_i, E_j$ are pairwise disjoint and ii) $\cup_{i=1}^\infty E_i = S$, then
\begin{equation*}
     P(A) = \sum_{i=1}^\infty P(A\cap E_i) = \sum_{i=1}^\infty P(A \mid E_i) P(E_i)
\end{equation*}


\noindent The \textbf{continuity of probability measures} state
\begin{align*}
    (i) \;& E_1 \subset E_2 \subset \dots \textrm{   Let } E_\infty = \cup_i E_i \textrm{, then } P(E_n) \longrightarrow P(E_\infty) \textrm{ as } n \longrightarrow \infty\\
    (ii) \;& E_1 \supset E_2 \supset \dots \textrm{   Let } E_\infty = \cap_i E_i \textrm{, then } P(E_n) \longrightarrow P(E_\infty) \textrm{ as } n \longrightarrow \infty\\\\
\end{align*}

\subsubsection{Conditional probability}
The conditional probability is the probability of one event occurring, given the other event occurring. A reframing of conditional probability (see formula below) is the probability of both events occurring, divided by the marginal probability of one of the events occurring. 
\begin{align*}
    p_{X|Y}(x|y) = \frac{p(x,y)}{p_y(y)}
\end{align*}
\textbf{Bayes Theorem} leverages conditional probabilities of measured events to glean conditional probabilities of unmeasured events:
\begin{equation*}
    P(E_i \mid B) = \frac{P(B \mid E_i)P(E_i)}{\sum_{j=1}^\infty P(B \mid E_j)P(E_j)} = \frac{P(B \mid E_i)P(E_i)}{P(B)}
\end{equation*}
Where $E_1, E_2, \dots$ form a partition of the sample space.

\subsubsection{Independence}
Events $A$ and $B$ are independent if $P(A\cap B) = P(A)P(B)$

It is possible for events to be pairwise independent, but not mutually independent. For example, toss a pair of dice and let $D_1$ be the number for die 1 and $D_2$ be the number for die 2. Define $E_i = \{D_i \leq 2\}$. And define $E_3 = \{ 3 \leq \max(D_1, D_2) \leq 4 \}$. These events are pairwise independent, but $P(E_1\cap E_2 \cap E_3) = 0$, so they are not mutually independent. 

% RANDOM VARIABLES
\section{Random variables and expected value}
\textbf{Random variables} are functions connecting a sample space to real numbers. They are formally defined as
\begin{equation*}
    \{ \omega \in S : X(\omega) \leq t \} \in \mathcal{E}
\end{equation*}
For example, if coin tosses produce a sample space of {Heads, Tails}, a random variable can be the number of heads. 

% DISTRIBUTION FUNCTIONS
\subsection{Discrete distribution functions}
\subsubsection{Bernoulli}
\textbf{Probability mass function ($Bernouli(p)$):} TODO
\begin{align*}
    P(X) = p^x(1-p)^{1-x}
\end{align*}
\textbf{Expected value:} $p$
\textbf{Variance:} $p(1-p)$

\subsubsection{Binomial distribution}
\textbf{Probability mass function ($Bin(n,p)$):} For random variable $X$, the number of successes in $n$ trials, the probability of observing $j$ successes where each success has probability $p$ is
\begin{equation*}
    P(X = j) = {n \choose j} p^j (1 - p)^{n-j}
\end{equation*}
\textbf{Expected value:} $np$
\textbf{Variance:} $np(1-p)$

\subsubsection{Geometric distribution}
\textbf{Probability mass function ($Geom(p)$):} For random variable $X$, the number of trials until the first success (included) with probability $p$ is
\begin{equation*}
    P(X=j) = (1-p)^{j-1}p
\end{equation*}
\textbf{Expected value:} $\frac{1}{p}$
\textbf{Variance:} $\frac{1-p}{p}$

\subsubsection{Negative binomial}
\textbf{Probability mass function ($NB(r, p)$):} TODO
\begin{equation*}
	P(X = j) = {k + r - 1 \choose k} (1-p)^rp^k
\end{equation*}
\textbf{Expected value:} $\frac{pr}{1-p}$
\textbf{Variance:} $\frac{pr}{(1 - p)^2}$

\subsubsection{Poisson distribution}
\textbf{Probability mass function ($Pois(\lambda)$):} TODO
\begin{equation*}
	\frac{\lambda^ke^{-\lambda}}{k!}
\end{equation*}
\textbf{Expected value:} $\lambda$
\textbf{Variance:} $\lambda$

\subsubsection{Hypergeometric distribution}
\textbf{Probability mass function ($todo$):} TODO
\begin{equation*}
\end{equation*}
\textbf{Expected value:} $todo$
\textbf{Variance:} $todo$


\subsection{Continuous distribution functions}
\subsubsection{Uniform distribution}
\textbf{Probability density function ($todo$):}
\begin{equation*}
\end{equation*}
\textbf{Expected value:} $todo$
\textbf{Variance:} $todo$

\subsubsection{Normal distribution}
\textbf{Probability density function ($todo$):}
\begin{equation*}
\end{equation*}
\textbf{Expected value:} $todo$
\textbf{Variance:} $todo$

\subsubsection{Exponential distribution}
\textbf{Probability density function ($todo$):}
\begin{equation*}
\end{equation*}
\textbf{Expected value:} $todo$
\textbf{Variance:} $todo$

\subsubsection{Gamma distribution}
\textbf{Probability density function ($todo$):}
\begin{equation*}
\end{equation*}
\textbf{Expected value:} $todo$
\textbf{Variance:} $todo$

% MARGINAL, JOINT, AND CONDITIONAL DISTRIBUTIONS
\section{Marginal, joint, and conditional distributions}

% JOINT DISTRIBUTIONS
\subsection{Joint distributions}
The cumulative density function (cdf) and probability mass function (pmf) satisfy respectively
\begin{align*}
    \textrm{cdf: } F_{X_1,\dots,X_n}(x_1, \dots, x_n) = P(X_i \leq x_1, \dots, X_n \leq x_n)\\
    \textrm{pmf: } f_{X_1,\dots,X_n}(x_1, \dots, x_n) = P(X_1=x_1, \dots, X_n = x_n)
\end{align*}
The joint density function $f$ then satisfies, for $E \subset \mathbb{R}^n$,
\begin{equation*}
    P((X_1, \dots, X_n)\in E) = \int \dots \int_E f_{X_1,\dots,X_n}dx_1\dots dx_n
\end{equation*}
When random variables are independent, the joint cdf and pmf satisfy respectively
\begin{align*}
    \textrm{cdf: } & P(X_1\leq x_1, \dots, X_n \leq x_n) = P(X_1\leq x_1)\dots P(X_n\leq x_n)\\
    \textrm{pmf: } & P(X_1= x_1, \dots, X_n = x_n) = P(X_1=x_1)\dots P(X_n=x_n)\\
\end{align*}

\subsubsection{Distribution of sums of independent random variables}
The following combination of marginal distributions is called a \textbf{convolution}.\\
If $X$ and $Y$ have densities, the cdf of $X+Y$ is
\begin{align*}
    F_{X+Y}(t) &= P(X+Y\leq t)\\
    &= P(X \leq t-y)\\
    &= \int_{-\infty}^\infty P(X \leq t-y \mid Y=y) f_x(y)dy \textrm{, to get marginal distribution}\\
    &= \int_{-\infty}^\infty F_x(X \leq t-y) f_x(y)dy \textrm{, since $X,Y$ independent}
\end{align*}
Likewise, the density of the sum is
\begin{equation*}
    f_{X+Y}(t) = \int_{-\infty}^\infty f_x(X \leq t-y) f_x(y)dy
\end{equation*}

\subsubsection{Expectation of joint distributions}
For $X,Y$ joint distribution, $f_{X,Y}(x,y)$, or probability mass function, $p(x,y)$
\begin{align*}
    \textrm{pmf: } E[g(X,Y)] &= \sum_s g(X(s), Y(s))p(s)\\
    &= \sum_x\sum_y g(x,y) \sum_{s:X(s)=x,Y(s)=y}p(s)\\
    &= \sum_x\sum_y g(x,y)p(x,y)\\ \\
    \textrm{pdf: } E[g(X,Y)] &= \int_{y= -\infty}^{\infty} \int_{x= -\infty}^{\infty} g(x,y)f(x,y)dxdy
\end{align*}

% MARGINAL DISTRIBUTIONS
\subsection{Marginal distributions}
Marginal density functions or marginal probability mass functions are obtained by integrating or summing out the other variables
\begin{equation*}
    f_Y(y) = \sum_x y P(Y = y \mid x)
\end{equation*}

% CONDITIONAL DISTRIBUTIONS
\subsection{Conditional distributions}
\textrm{Reminder: }
\begin{align*}
    p_{X|Y}(x|y) = \frac{p(x,y)}{p_y(y)}
\end{align*}
We can use conditional probabilities to restate the \textbf{law of total probability}:
\begin{equation*}
    P(E) = \int_{-\infty}^\infty P(E \mid X=x)f(x)dx
\end{equation*}




%EXPECTED VARIABLES
\section{Expected variables}
\subsection{Expected value}
The expected value (or mean) of a discrete random variable, $X$, is
\begin{equation*}
    E(X) = \sum_x xP(X=x)
\end{equation*}
Which can also be written as
\begin{align*}
    E(X) &= \sum_{x \in S} X(s)p(s) \textrm{, where $p(s)$ is the probability that element $s \in S$ occurs:}\\
    \textrm{Proof:}&\\
    E(X) &= \sum_i x_iP(X=x_i) \textrm{, for } E_i = \{X = x_i\} = \{s \in S : X(s) = x_i\}\\
    &= \sum_i x_i \sum_{s\in E_i} p(s) = \sum_i \sum_{s\in E_i} x_i p(s)\\
    &= \sum_i \sum_{s\in E_i} X(s) p(s) = \sum_{s\in S} x_i p(s)
\end{align*}
This latter equation structure helps build intuition about the linearity of the expected value function and allows us to derive several other properties of expected values. In the general case:
\begin{align*}
    E(g(X)) &= \sum_i g(x_i)p_X(x_i) \textrm{, assuming } g(x_i) = y_i\\
    \textrm{Proof:}&\\
    \sum_i g(x_i)p_X(x_i) &= \sum_j \sum_{i:g(x_i)=y_j} g(x_i) p_X(x_i) = \sum_j \sum_{i:g(x_i)=y_j} y_j p_X(x_i) \\
    &= \sum_j y_j \sum_{i:g(x_i)=y_j} p_X(x_i) = \sum_j y_j P(g(X) = x_i)\\
    &= E(g(X))
\end{align*}
And from this general equation we can get two key properties of the expected value:
\begin{align*}
    (i) \; & E(aX + b) = aE(X) + b\\
    & E(aX + b) = \sum_{x\in S} (aX(s) + b) p(s) = a\sum_{s\in S}X(s)p(s) + \sum_{s\in S}bp(s) = aE(X) + b\\ \\
    (ii) \; & E(X + Y) = E(X) + E(Y)\\
    & E(X + Y) = \sum_{s \in S} (X(s) + Y(s))p(s) = \sum_{s \in S} X(s)p(s) + \sum_{s \in S} Y(s)p(s) = E(X) + E(Y)
\end{align*}


\section{Variance, covariance, and correlation}
The \textbf{variance} of X is defined in relation to $E(X) = \mu$ as the expected value of the squared difference between the random variable the mean. The standard deviation, $\sigma$ is defined as the square root of the variance.
\begin{align*}
    Var(X) &= E((X - \mu)^2) = \sigma^2\\
    SD &= \sqrt{Var(X)} = \sqrt{\sigma^2} = \sigma
\end{align*}
Several properties of variance follow from linearity of expectation:
\begin{align*}
    (i) \; & Var(X) = E(X^2) - \mu^2\\
    & Var(X) = E((X - \mu)^2) = E(X^2 - 2X\mu + \mu^2) = E(X^2 - 2\mu X + \mu^2) \\
    & Var(X) = E(X^2) - 2\mu^2 + \mu^2 = E(X^2) - \mu^2\\ \\
    (ii) \; & Var(aX+b) = a^2Var(X) \\
    & Var(aX + b) = E((aX + b)^2) - E(aX + b)^2 = E(a^2X^2 + 2abX + b^2) - (aE(X)+b)^2\\
    & Var(aX + b) =a^2E(X^2) + 2abE(X) + b^2 - a^2E(X)^2 - 2abE(X) - b^2 = a^2E(X^2) - a^2E(X)^2 = a^2(E(X^2) - E(X)^2)
\end{align*}


% MOMENT GENERATING FUNCTIONS
\section{Moment generating functions}
The moment generating function of a random variable X is defined as 
\begin{equation*}
	M_X(t) = \mathbb{E}[e^{tX}] = \sum_{n=0}^\infty\frac{\mathbb{E}[X^n]}{n!}t^n
\end{equation*}
Notice its called a moment generating function because each derivative of this function can generate a new moment of $X$ at $t=0$:
\begin{equation*}
	M_X^{(n)}(0) = \mathbb{E}[X^n]
\end{equation*}
\subsection{Common MGF derivations}

% CONVERGENCE AND LIMIT THEOREMS
\section{Convergence and limit theorems}
\subsection{Convergence in probability}
\subsection{Convergence in $L_p$}
\subsection{Convergence in distribution}

\subsection{Law of large numbers}
For $X_1, X_2, \dots, X_n$ a sequence of i.i.d. random variables with $E(X_i) = \mu$,  $Var(X_i) = \sigma^2$, $\overline{X}_n = \frac{1}{n}\sum_{I = 1}^n X_i$, then for any $\epsilon > 0$
\begin{equation*}
	P(\abs{\overline{X}_n - \mu} > \epsilon) \longrightarrow 0 \textrm{ as } n \rightarrow \infty
\end{equation*}
\textbf{Proof:}
\begin{align*}
	\textrm{First find $\mathbb{E}(\overline{X}_n)$ and $Var(\overline{X}_n)$} \\
	& \mathbb{E}(\overline{X}_n) = \frac{1}{n}\sum_{I = 1}^n \mathbb{E}(X_i) = \mu \\
	& Var(\overline{X}_n) = \frac{1}{n^2}\sum_{I = 1}^nVar(X_i) = \frac{\sigma^2}{n} \textrm{, since $X_i$ independent}
\end{align*}
The desired result now follows immediately from Chebyshev’s inequality, which states
\begin{align*}
	& P(\abs{\overline{X}_n - \mu} > \epsilon) \leq \frac{Var(\overline{X}_n)}{\epsilon^2} = \frac{\sigma^2}{n\epsilon^2} \rightarrow 0 \textrm{ as } n \rightarrow \infty
\end{align*}

\subsection{Central limit theorem}
Most useful form of CLT:
\begin{align*}
	\sqrt{n}\frac{(\overline{X}_n - \mu)}{\sigma} &\longrightarrow N(0, 1)\\
	\sqrt{n}(\overline{X}_n - \mu) &\longrightarrow N(0, \sigma^2)
\end{align*}
\textbf{More formal definition and proof:} For $X_1, X_2, \dots, X_n$ a sequence of i.i.d. random variables with $E(X_i) = 0$,  $Var(X_i) = \sigma^2$, and the common cumulative distribution function $F$ and moment-generating function $M$ defined in a neighborhood of zero. Then
\begin{align*}
	\textrm{For } S_n = \sum_{i=1}^nX_i\\
	\lim_{n \rightarrow \infty}P(\frac{S_n}{\sigma \sqrt{n}} \leq x) = \Phi(x)
\end{align*}
\textbf{Proof:} Let $Z_n  = \frac{S_n}{\sigma \sqrt{n}}$. We show the MGF of $Z_n$ tends to the MGF of the standard normal distribution. Since $S_n$ is a sum of independent random variables,
\begin{align*}
	M_{S_n}(t) &= [M(t)]^n \textrm{ and } M_{Z_n}(t) = [M(\frac{t}{\sigma \sqrt{n}})]^n\\
	\textrm{Reminder: } & \textrm{Taylor series expansion of } M(s) = M(0)+sM'(0)+ \frac{1}{2}sM''(0) + \epsilon_s \\
	M(\frac{t}{\sigma\sqrt{n}}) &= 1 + \frac{1}{2}\sigma^2(\frac{t}{\sigma \sqrt{n}})^2 + \epsilon_n \textrm{ with } E(X) = M'(0) = 0, Var(X) = M''(0) = \sigma^2\\
	M_{Z_n}(t) &= (1 + \frac{t^2}{2n} + \epsilon_n)^n\\
	M_{Z_n}(t) &\longrightarrow e^{\frac{t^2}{2}} \textrm{ as } n \longrightarrow \infty \textrm{, by the infinite series convergence to $e^a$}
\end{align*}
Since $e^{\frac{t^2}{2}}$ is the MGF of the standard normal distribution, we have proven the central limit theorem.







\end{document}
