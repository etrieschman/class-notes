\documentclass{article}
\usepackage[utf8]{inputenc}
\usepackage{mathtools,amssymb}
\usepackage{amsmath}
\usepackage{hyperref}
\usepackage{lipsum}
\usepackage{layout}
\usepackage{geometry}
\newcommand*{\vertbar}[1][10ex]{\rule[-1ex]{0.5pt}{#1}}
\newcommand*{\horzbar}{\rule[.5ex]{2.5ex}{0.5pt}}
\newcommand{\norm}[2]{\left\lVert#1\right\rVert_#2}
\newcommand{\abs}[1]{\lvert#1\rvert}

\title{STATS200 class notes}
\author{Erich Trieschman}
\date{2021 Fall quarter}

\newcommand{\userMarginInMm}{20}
\geometry{
%  legal,
 left=\userMarginInMm mm,
 right=\userMarginInMm mm,
 top=\userMarginInMm mm,
 bottom=20mm,
 footskip=5mm}

 \hypersetup{
    colorlinks=true,
    linkcolor=blue,
    filecolor=magenta,      
    urlcolor=cyan,
    pdfpagemode=FullScreen,
    }

\begin{document}
\maketitle

\tableofcontents

% COMBINATORICS AND PROBABILITY REVIEW
\section{Review: Combinatorics and probability}
%Logs, derivatives, integration, taylor expansions
\subsection{Calculus cheat sheet}
\subsubsection{Logs}
\begin{itemize}
	\item $log_b(M * N) = Lob_bM + log_bN$
	\item $log_b(\frac{M}{N}) = log_bM - log_bN$
	\item $log_b(M^k) = klog_bM$
	\item $e^ne^m = e^{n+m}$
\end{itemize}
\subsubsection{Derivatives}
\begin{itemize}
	\item $(x^n)' = nx^{n-1}$
	\item $(e^x)' = e^x$
	\item $(e^{u(x)})' = u'(x)e^x$
	\item $(log_e(x))' = (lnx)' = \frac{1}{x}$
	\item $(f(g(x)))' = f'(g(x))g'(x)$
\end{itemize}
\subsubsection{Integrals}
\begin{itemize}
	\item TODO: Integration by parts
\end{itemize}
\subsubsection{Infinite series and sums}
\begin{itemize}
	\item $e^x = 1 + x + \frac{x^2}{2!} + \frac{x^3}{3!} + \dots = \sum_{n=0}^\infty \frac{x^n}{n!}$
	\item $\frac{1}{1-x} = 1 + x + x^2 + x^3 + \dots = \sum_{n=0}^\infty a^x$ for $\abs{x} < 1$
	\item $ln(1 + x) = 1 - x + \frac{x^2}{2} - \frac{x^3}{3} + \dots = \sum_{n=0}^\infty (-1)^n\frac{x^n}{n}$
	\item $(1 + \frac{a}{n})^n \longrightarrow e^a$
\end{itemize}

% EVENTS AND SETS
\subsection{Events and sets}
Set operations follow commutative, associative, and distributive laws:
\begin{itemize}
    \item Commutative: $E \cup F = F \cup E$ and $E\cap F = F \cap E $ (also written $EF = FE$)
    \item Associative: $(E\cup F)\cup G = E \cup (f\cup G)$ and $(E\cap F)\cap G = E\cap(F\cap G)$
    \item Distributive: $(E\cup F)\cap G = (E\cap G) \cup (F \cap G) = E\cap G \cup F \cap G$ and $E\cap F\cup G = (E\cup G) \cap (F \cup G) = E\cup G \cap F \cup G$
\end{itemize}
\textbf{DeMorgan's Laws} relate the complement of a union to the intersection of complements:
\begin{itemize}
    \item $(\cup_{i=1}^n E_i)^c = \cap_{i=1}^nE_i^c$
    \item $(\cap_{i=1}^n E_i)^c = \cup_{i=1}^nE_i^c$
\end{itemize}

% PROBABILITY
\subsection{Probability}
A \textbf{probability space} is defined by a triple of objects $(S, \mathcal{E}, P)$:
\begin{itemize}
    \item $S:$ Sample space
    \item $\mathcal{E}:$ Set of possible events within the sample space. Set of events are assumed to be $\theta$-field (below)
    \item $P:$ Probability for each event
\end{itemize}
A \textbf{$\theta$-field} is a collection of subsets $\mathcal{E} \subset S$ that satisfy
\begin{itemize}
    \item $0 \in \mathcal{E}$
    \item $E \in \mathcal{E} \Rightarrow E^C \in \mathcal{E}$
    \item $E_i \in \mathcal{E}$ for {$1, 2, \dots$} $\Rightarrow \cup_{i=1}^\infty E_i \in \mathcal{E}$
\end{itemize}
\textbf{Basic probability properties}
\begin{itemize}
	\item $P(A^C)=1 - P(A)$
	\item $P(0)=0$
	\item $A\subset B \longrightarrow P(A) \leq P(B)$
	\item $P(A \cup B)=P(A)+P(B) - P(A \cap B)$ 
\end{itemize}
The \textbf{law of total probability} relates marginal probabilities to conditional probabilities. For a partition, {$E_1, E_2, \dots$} of set, $S$, where a partition implies i) $E_i, E_j$ are pairwise disjoint and ii) $\cup_{i=1}^\infty E_i = S$, then
\begin{equation*}
     P(A) = \sum_{i=1}^\infty P(A\cap E_i) = \sum_{i=1}^\infty P(A \mid E_i) P(E_i)
\end{equation*}


\noindent The \textbf{continuity of probability measures} state
\begin{align*}
    (i) \;& E_1 \subset E_2 \subset \dots \textrm{   Let } E_\infty = \cup_i E_i \textrm{, then } P(E_n) \longrightarrow P(E_\infty) \textrm{ as } n \longrightarrow \infty\\
    (ii) \;& E_1 \supset E_2 \supset \dots \textrm{   Let } E_\infty = \cap_i E_i \textrm{, then } P(E_n) \longrightarrow P(E_\infty) \textrm{ as } n \longrightarrow \infty\\\\
\end{align*}

\subsubsection{Conditional probability}
The conditional probability is the probability of one event occurring, given the other event occurring. A reframing of conditional probability (see formula below) is the probability of both events occurring, divided by the marginal probability of one of the events occurring. 
\begin{align*}
    p_{X|Y}(x|y) = \frac{p(x,y)}{p_y(y)}
\end{align*}
\textbf{Bayes Theorem} leverages conditional probabilities of measured events to glean conditional probabilities of unmeasured events:
\begin{equation*}
    P(E_i \mid B) = \frac{P(B \mid E_i)P(E_i)}{\sum_{j=1}^\infty P(B \mid E_j)P(E_j)} = \frac{P(B \mid E_i)P(E_i)}{P(B)}
\end{equation*}
Where $E_1, E_2, \dots$ form a partition of the sample space.

\subsubsection{Independence}
Events $A$ and $B$ are independent if $P(A\cap B) = P(A)P(B)$

It is possible for events to be pairwise independent, but not mutually independent. For example, toss a pair of dice and let $D_1$ be the number for die 1 and $D_2$ be the number for die 2. Define $E_i = \{D_i \leq 2\}$. And define $E_3 = \{ 3 \leq \max(D_1, D_2) \leq 4 \}$. These events are pairwise independent, but $P(E_1\cap E_2 \cap E_3) = 0$, so they are not mutually independent. 

% RANDOM VARIABLES
\section{Random variables and common distribution functions}
\textbf{Random variables} are functions connecting a sample space to real numbers. They are formally defined as
\begin{equation*}
    \{ \omega \in S : X(\omega) \leq t \} \in \mathcal{E}
\end{equation*}
For example, if coin tosses produce a sample space of \{Heads, Tails\}, a random variable can be the number of heads. 

% DISTRIBUTION FUNCTIONS
\subsection{Discrete distribution functions}
\subsubsection{Bernoulli}
\textbf{Probability mass function ($Bernouli(p)$):} Random variable $X$ takes the value 1 with probability $p$ and the value 0 with probability $1-p$
\begin{align*}
    p(x) = p^x(1-p)^{1-x} \;,\; x \in \{0, 1\}
\end{align*}
\textbf{Expected value:} $p$\\
\textbf{Variance:} $p(1-p)$

\subsubsection{Binomial distribution}
\textbf{Probability mass function ($Bin(n,p)$):} For random variable $X$, the number of successes in $n$ trials, the probability of observing $j$ successes where each success has probability $p$ is
\begin{equation*}
    P(X = j) = {n \choose j} p^j (1 - p)^{n-j}
\end{equation*}
\textbf{Expected value:} $np$\\
\textbf{Variance:} $np(1-p)$

\subsubsection{Geometric distribution}
\textbf{Probability mass function ($Geom(p)$):} For random variable $X$, the number of trials until the first success (included) with probability $p$ is
\begin{equation*}
    P(X=j) = (1-p)^{j-1}p
\end{equation*}
\textbf{Expected value:} $\frac{1}{p}$\\
\textbf{Variance:} $\frac{1-p}{p}$

\subsubsection{Negative binomial}
\textbf{Probability mass function ($NB(r, p)$):} For random variable $X$, the number of successes, $k$ before a specified number of failures, $r$, with probability of success $p$ is
\begin{equation*}
	P(X = k) = {k + r - 1 \choose k} (1-p)^rp^k
\end{equation*}
\textbf{Expected value:} $\frac{pr}{1-p}$\\
\textbf{Variance:} $\frac{pr}{(1 - p)^2}$

\subsubsection{Poisson distribution}
\textbf{Probability mass function ($Pois(\lambda)$):} For random variable, $X$, the number of events, $k$, occurring in a fixed interval of time or space if these events occur with a known constant mean rate, $\lambda$
\begin{equation*}
	P(X = k) = \frac{\lambda^ke^{-\lambda}}{k!}
\end{equation*}
\textbf{Expected value:} $\lambda$\\
\textbf{Variance:} $\lambda$


\subsection{Continuous distribution functions}
\subsubsection{Uniform distribution}
\textbf{$Unif(a, b)$:} The distribution describes an experiment where there is an arbitrary outcome that lies between certain bounds.[1] The bounds are defined by the parameters, a and b, which are the minimum and maximum values
\begin{align*}
    pdf: &\; f(x) = \begin{cases}
        \frac{1}{b-a} & \textrm{ for } x \in [a,b]\\
        0 & \textrm{ otherwise}
    \end{cases}\\
    cdf: &\ F(x) = \begin{cases}
        0 & \textrm{ for } x < a\\
        \frac{x-a}{b-a} & \textrm{ for } x \in [a,b]\\
        1 & \textrm{ for } x > b
    \end{cases}; 
\end{align*}
\textbf{Expected value:} $\frac{1}{2}(a + b)$\\
\textbf{Variance:} $\frac{1}{12}(b - a)^2$

\subsubsection{Normal distribution}
\textbf{$N(\mu, \sigma)$}
\begin{align*}
    pdf: & \; f(x) = \frac{1}{\sigma \sqrt{2 \pi}} \exp \left (-\frac{(x - \mu)^2}{2 \sigma ^ 2}\right )\\
    cdf: & \; F(x) = \frac{1}{\sqrt{2 \pi}} \int_{- \infty}^x e^{-t^2 / 2} dt
\end{align*}
\textbf{Expected value:} $\mu$\\
\textbf{Variance:} $\sigma^2$

\subsubsection{Exponential distribution}
\textbf{$Exp(\lambda)$:} the probability distribution of the time between events in a Poisson point process, i.e., a process in which events occur continuously and independently at a constant average rate. It is a particular case of the gamma distribution.
\begin{align*}
    pdf: & \; f(x) = \lambda e^{-\lambda x}\\
    cdf: & \; F(x) = 1 - e^{-\lambda x}
\end{align*}
\textbf{Expected value:} $\frac{1}{\lambda}$\\
\textbf{Variance:} $\frac{1}{\lambda^2}$

\subsubsection{Gamma distribution}
\textbf{$Gamma(\alpha, \lambda)$:} a two-parameter family of continuous probability distributions.
\begin{align*}
    pdf: & \; f(x) = \frac{\lambda^{\alpha}}{\Gamma(\alpha)}x^{\alpha-1} e^{-\lambda x} \; \textrm{, where } \Gamma(\alpha) = (\alpha - 1)! \textrm{ for any positive integer, } \alpha\\
    cdf: & \; F(x) =  \frac{1}{\Gamma(\alpha)}\gamma(\alpha, \lambda x) \textrm{, where } \gamma(\alpha, x) = \int_0^x t^{\alpha - 1}e^{-t}dt
\end{align*}
\textbf{Expected value:} $\frac{\alpha}{\lambda}$\\
\textbf{Variance:} $\frac{\alpha}{\lambda^2}$

\subsubsection{Cauchy distribution}
\textbf{$Cauchy(t, s)$:} The Cauchy distribution is often used in statistics as the canonical example of a "pathological" distribution since both its expected value and its variance are undefined
\begin{align*}
    pdf: & \; f(x) = \frac{1}{s \pi (1 + (x - t)/s)^2)} \textrm{, where } s \textrm{ is the scale parameter and } t \textrm{ is the location parameter}\\
    cdf: & \; \frac{1}{\pi} \arctan \left ( \frac{x - t}{s} \right ) + \frac{1}{2}
\end{align*}
\textbf{Expected value:} $DNE$\\
\textbf{Variance:} $DNE$

\subsubsection{Beta distribution}
\textbf{$Beta(\alpha, \beta)$:} a family of continuous probability distributions defined on the interval [0, 1] parameterized by two positive shape parameters that appear as exponents of the random variable and control the shape of the distribution.
\begin{align*}
    pdf: & \; f(x) = \frac{\Gamma(\alpha + \beta)}{\Gamma(\alpha) \Gamma(\beta)} x^{\alpha - 1} (1-x)^{\beta - 1} \textrm{, where } x \in [0, 1] \textrm{, and } \Gamma(k) = (k - 1)! \textrm{ for any positive integer } k\\
\end{align*}
\textbf{Expected value:} $\frac{\alpha}{\alpha + \beta}$\\
\textbf{Variance:} $\frac{\alpha \beta}{(\alpha + \beta)^2(\alpha + \beta + 1)}$

% MARGINAL, JOINT, AND CONDITIONAL DISTRIBUTIONS
\section{Joint, marginal, and conditional distributions}

% JOINT DISTRIBUTIONS
\subsection{Joint distributions}
The cumulative density function (cdf) and probability mass function (pmf) satisfy respectively
\begin{align*}
    \textrm{cdf: } F_{X_1,\dots,X_n}(x_1, \dots, x_n) = P(X_i \leq x_1, \dots, X_n \leq x_n)\\
    \textrm{pmf: } f_{X_1,\dots,X_n}(x_1, \dots, x_n) = P(X_1=x_1, \dots, X_n = x_n)
\end{align*}
The joint density function $f$ then satisfies, for $E \subset \mathbb{R}^n$,
\begin{equation*}
    P((X_1, \dots, X_n)\in E) = \int \dots \int_E f_{X_1,\dots,X_n}dx_1\dots dx_n
\end{equation*}
When random variables are independent, the joint cdf and pmf satisfy respectively
\begin{align*}
    \textrm{cdf: } & P(X_1\leq x_1, \dots, X_n \leq x_n) = P(X_1\leq x_1)\dots P(X_n\leq x_n) = \prod_{i=1}^n P(X_i \leq x_i)\\
    \textrm{pmf: } & P(X_1= x_1, \dots, X_n = x_n) = P(X_1=x_1)\dots P(X_n=x_n) = \prod_{i=1}^nP(X_i = x_i)\\
\end{align*}

\subsubsection{Distribution of sums of independent random variables}
The following combination of marginal distributions is called a \textbf{convolution}.\\
If $X$ and $Y$ have densities, the cdf of $X+Y$ is
\begin{align*}
    F_{X+Y}(t) &= P(X+Y\leq t)\\
    &= P(X \leq t-y)\\
    &= \int_{-\infty}^\infty P(X \leq t-y \mid Y=y) f_x(y)dy \textrm{, to get marginal distribution}\\
    &= \int_{-\infty}^\infty F_x(t-y) f_x(y)dy \textrm{, since $X,Y$ independent}
\end{align*}
Likewise, the density of the sum is
\begin{equation*}
    f_{X+Y}(t) = \int_{-\infty}^\infty f_x(t-y) f_x(y)dy
\end{equation*}
And similarly for discrete random variables
\begin{equation*}
    p_{X + Y}(t) = \sum_{x=-\infty}^{\infty} p_Y(t-y) p_X(y)
\end{equation*}

\subsubsection{Expectation of joint distributions}
For $X,Y$ joint distribution, $f_{X,Y}(x,y)$, or probability mass function, $p(x,y)$
\begin{align*}
    \textrm{pmf: } E[g(X,Y)] &= \sum_s g(X(s), Y(s))p(s)\\
    &= \sum_x\sum_y g(x,y) \sum_{s:X(s)=x,Y(s)=y}p(s)\\
    &= \sum_x\sum_y g(x,y)p(x,y)\\ \\
    \textrm{pdf: } E[g(X,Y)] &= \int_{y= -\infty}^{\infty} \int_{x= -\infty}^{\infty} g(x,y)f(x,y)dxdy
\end{align*}

% MARGINAL DISTRIBUTIONS
\subsection{Marginal distributions}
Marginal density functions or marginal probability mass functions are obtained by integrating or summing out the other variables
\begin{align*}
    pmf: &\;  f_Y(y) = \sum_x y P(Y = y \mid x)
    pdf: &\; f_Y(y) = \int_a^b f(x,y)dx \textrm{, where } x \in [a,b]
\end{align*}

% CONDITIONAL DISTRIBUTIONS
\subsection{Conditional distributions}
\textrm{Reminder: }
\begin{align*}
    p_{X|Y}(x|y) = \frac{p(x,y)}{p_y(y)} \; \textrm{ and } \; f_{X\mid Y}(x \mid y) = \frac{f_{X,Y(x, y)}}{f_X(x)}
\end{align*}
We can use conditional probabilities to restate the \textbf{law of total probability}:
\begin{equation*}
    P(E) = \int_{-\infty}^\infty P(E \mid X=x)f(x)dx
\end{equation*}




%EXPECTED VARIABLES
\section{Expected variables}
\subsection{Expected value}
\begin{equation*}
    E(X) = \sum_x xP(X=x)
\end{equation*}
Which can also be written as
\begin{align*}
    E(X) &= \sum_{x \in S} X(s)p(s) \textrm{, where $p(s)$ is the probability that element $s \in S$ occurs:}\\
    \textrm{Proof:}&\\
    E(X) &= \sum_i x_iP(X=x_i) \textrm{, for } E_i = \{X = x_i\} = \{s \in S : X(s) = x_i\}\\
    &= \sum_i x_i \sum_{s\in E_i} p(s) = \sum_i \sum_{s\in E_i} x_i p(s)\\
    &= \sum_i \sum_{s\in E_i} X(s) p(s) = \sum_{s\in S} x_i p(s)
\end{align*}
This equation structure helps proof several properties of the expected value:\\
\begin{itemize}
    \item $E(g(X)) = \sum_i g(x_i)p_X(x_i) \textrm{, assuming } g(x_i) = y_i$ 
\begin{align*}
    \textrm{Proof:}&\\
    \sum_i g(x_i)p_X(x_i) &= \sum_j \sum_{i:g(x_i)=y_j} g(x_i) p_X(x_i) = \sum_j \sum_{i:g(x_i)=y_j} y_j p_X(x_i) \\
    &= \sum_j y_j \sum_{i:g(x_i)=y_j} p_X(x_i) = \sum_j y_j P(g(X) = x_i)\\
    &= E(g(X))
\end{align*}
\item $E(aX + b) = aE(X) + b$
\begin{align*}
    E(aX + b) = \sum_{s\in S} (aX(s) + b) p(s) = a\sum_{s\in S}X(s)p(s) + \sum_{s\in S}bp(s) = aE(X) + b
\end{align*}
\item $E(X + Y) = E(X) + E(Y)$
\begin{align*}
    E(X + Y) = \sum_{s \in S} (X(s) + Y(s))p(s) = \sum_{s \in S} X(s)p(s) + \sum_{s \in S} Y(s)p(s) = E(X) + E(Y)
\end{align*}
\end{itemize}


\subsection{Variance}
\begin{align*}
    Var(X) &= E((X - E(X)))^2) = \sigma^2\\
    SD &= \sqrt{Var(X)} = \sqrt{\sigma^2} = \sigma
\end{align*}
Several properties of variance follow from linearity of expectation:
\begin{align*}
    (i) \; & Var(X) = E(X^2) - \mu^2\\
    & Var(X) = E((X - \mu)^2) = E(X^2 - 2X\mu + \mu^2) = E(X^2 - 2\mu X + \mu^2) \\
    & Var(X) = E(X^2) - 2\mu^2 + \mu^2 = E(X^2) - \mu^2\\ \\
    (ii) \; & Var(aX+b) = a^2Var(X) \\
    & Var(aX + b) = E((aX + b)^2) - E(aX + b)^2 = E(a^2X^2 + 2abX + b^2) - (aE(X)+b)^2\\
    & Var(aX + b) =a^2E(X^2) + 2abE(X) + b^2 - a^2E(X)^2 - 2abE(X) - b^2 = a^2E(X^2) - a^2E(X)^2 = a^2(E(X^2) - E(X)^2)\\ \\
    (iii) \; & Var(X + Y) = Var(X) + Var(Y) \textrm{ for $X,Y$ independent}\\
    & Var(X+Y) = E((X + Y)^2) - E(X + Y)^2 = E(X^2) + 2E(XY) + E(Y^2) - E(X^2) - 2E(X)E(Y) - E(Y)^2\\
    & Var(X+Y) = E(X^2) - E(X)^2 + E(Y^2) - E(Y)^2 \textrm{, since } E(XY) = 0 \textrm{ (by independence) and } E(X)=E(Y)=0 \textrm{ (WLOG)}\\
    &Var(X+Y) = Var(X) + Var(Y)
\end{align*}

\subsection{Covariance}
\begin{equation*}
    Cov(X, Y) = E((X - E(X)(Y - E(Y)) = E(XY) - E(X)E(Y)
\end{equation*}
Several properties of covariance follow from linearity of expectation
\begin{align*}
    (i) \; & Cov(X, X) = Var(X):\\
    & Cov(X, X) = E[(X - E(X)(X - E(X))] = E[(X - E(X))^2] = Var(X)\\ \\
    (ii) \; & Cov(X, Y) = E(XY) - E(X)E(Y):\\
    & Cov(X, Y) = E[(X - E(X)(Y - E(Y))] = E(XY - E(Y)X - E(X)Y + E(X)E(Y)) \\ 
    & Cov(X, Y) = E(XY) - E(X)E(Y) - E(X)E(Y) + E(X)E(Y) = E(XY) - E(X)E(Y)\\ \\
    (iii) \; & \textrm{if $X, Y$ independent, then} Cov(X, Y = 0)\\ \\
    (iv) \; & Cov(aX, bY) = abCov(X, Y):\\
    & Cov(aX, bY)= E(abXY) - E(aX)E(bY) = ab(E(XY) - E(X)E(Y)) = abCov(X, Y) \\ \\
    (v)\; & Cov(X, Y+Z) = Cov(X, Y) + Cov(X, Z):\\
    & Cov(X, Y+Z) = E(X(Y+Z)) - E(X)E(Y+Z) \\
    & Cov(X, Y+Z) = E(XY) + E(XZ) - E(X)E(Y) - E(X)E(Z) = Cov(X, Y) + Cov(X, Z)\\ \\
    (vi)\; & Cov(U, V) = \sum_i \sum_j b_id_jCov(X_i, Y_j) \textrm{, with } U = a + \sum_i b_iX_i \textrm{ and } V = c + \sum_j d_j Y_j:\\
    (vii)\; & Var(X + Y) = Var(X) + Var(Y) + 2Cov(X, Y):\\
    & Var (X + Y) = Cov(X+Y, X+Y) = Cov(U, V) \textrm{, for } U = V = X+Y \\
    & Var (X + Y) = Cov(U, V) = Cov(X, X) + Cov(X, Y) + Cov(Y, Y) + Cov(Y, X) \textrm{, using $vi$} \\
    & Var (X + Y) = Var(X) + Var(Y) + 2Cov(X, Y) 
\end{align*}

\subsection{Correlation}
\begin{equation*}
    \rho = \frac{Cov(X, Y)}{\sqrt{Var(X)Var(Y)}}
\end{equation*}

\subsection{Key theorems}
\subsubsection{Iterated expectation}
\textbf{Law of iterated expectation: } $E(E(Y\mid X)) = E(Y)$\\
\textbf{Proof: }
\begin{align*}
    E(Y\mid X) &= \sum_y y\frac{f_{X, Y}(X, y)}{f_X(X)}\\
    E(E(Y\mid X)) &= \sum_x \sum_y \left ( y\frac{f_{X, Y}(x, y)}{f_X(x)} \right ) f_X(x) = \sum_x \sum_y y f_{X, Y}(x, y) = \sum_y y f_{Y}(y) = E(Y)
\end{align*}

\subsubsection{Variance decomposition}
\textbf{Variance decomposition formula: } $Var(Y) = E(Var(Y\mid X)) + Var(E(Y \mid X))$\\

\subsubsection{Cauchy-Schwartz inequality}
\textbf{Cauchy-Schwartz inequality: }$E(UV)^2 \leq E(U^2)E(V^2) \textrm{, with equality if } P(cU=U) = 1 \textrm{ for some constant, } c$\\
\textbf{Proof:}
\begin{align*}
    \textrm{let } h(t) &= E((tU - V)^2) \geq 0\\
    h(t) &= t^2E(U^2) - 2tE(UV) + E(V^2) \textrm{, a quadradic equation}\\
    h(t) \geq 0 &\Rightarrow \textrm{discriminant} \leq 0\\
    &\Rightarrow 4E(UV)^2 - 4E(U^2)E(V^2) \leq 0 \\
    &\Rightarrow E(UV)^2 \leq E(U^2)E(V^2)\\
\end{align*}

\subsubsection{Jensen inequality}
\textbf{Jensen inequality: } $E(g(x)) \geq g(E(x))$ for $g(x)$ convex\\
\textbf{Proof: }
\begin{align*}
    \textrm{Let } E(X) = \mu \textrm{, and } L(X) \textrm{ a line s.t. } L(\mu) = g(E(x)):&\\
    &g(X) \geq L(X) \textrm{ for all } X\\
    &E(g(X)) \geq E(L(X)) = L(E(X)) = g(E(X))
\end{align*}

\subsubsection{Markov inequality}
\textbf{Markov inequality: } For $X\geq 0 \;, \;P(X \geq t) \leq \frac{E(X)}{t} \; \; \forall t>0$\\
\textbf{Proof: }
\begin{align*}
    \textrm{Let } y &= \begin{cases}
        1 & X \geq t\\
        0 & \textrm{otherwise}
    \end{cases}\\
    \textrm{Then } tY &\leq X \textrm{ since } \begin{cases}
        X \geq t & t*1 \leq X \\
        X < t & t*0 < X
    \end{cases}\\
    tY &\leq X \Longrightarrow E(tY) \leq E(X) \Longrightarrow tP(X \geq t) \leq E(X)) \Longrightarrow P(X \geq t) \leq \frac{E(X)}{t}
\end{align*}

\subsubsection{Chebyshev inequality}
\textbf{Chebyshev inequality: } $P(\abs{X - E(X)} \geq t) \leq \frac{Var(X)}{t^2} \; \; \forall t > 0$\\
\textbf{Proof: }
\begin{align*}
    P(\abs{X - E(X)} \geq t) &= P((X - E(X))^2 \geq t^2)\\
    P((X - E(X))^2 \geq t^2) &\leq \frac{E((X - E(X))^2)}{t^2} \textrm{, by Markov inequality}\\
    P((X - E(X))^2 \geq t^2) &\leq \frac{Var(X)}{t^2}
\end{align*}

\subsection{Moment generating function}
The moment generating function of a random variable X is defined as 
\begin{equation*}
	M_X(t) = \mathbb{E}[e^{tX}] = \sum_{n=0}^\infty\frac{\mathbb{E}[X^n]}{n!}t^n \textrm{ $\leftarrow$ power series}
\end{equation*}
Notice its called a moment generating function because each derivative of this function can generate a new moment of $X$ at $t=0$:
\begin{equation*}
	M_X^{(n)}(0) = \mathbb{E}[X^n]
\end{equation*}
\subsubsection{Common MGF derivations}
\begin{itemize}
    \item $Y = a+bX \Longrightarrow M_Y = e^{at}M_X(bt)$
    \item $Z = X+Y, X \perp Y \Longrightarrow M_Z = M_YM_X = E(e^tX)E(e^tY)$
\end{itemize}


% CONVERGENCE AND LIMIT THEOREMS
\section{Convergence and limit theorems}
\subsection{Convergence in probability}
A sequence of random variables, $X_n$, converges in probability, $X_n \overset{p}{\longrightarrow} X$ when
\begin{equation*}
    P(\abs{X_n - X} > \epsilon) \longrightarrow 0 \textrm{ as } n \longrightarrow \infty
\end{equation*}
\textbf{Consistent estimator:} An estimator, $T_n = T_n(X_1, \dots, X_n)$, which converges in probability to $g(\theta)$, a function of the model parameter\\\\
\textbf{Additional properties} of convergence in probability
\begin{itemize}
    \item if $X_n \overset{p}{\longrightarrow} X$ and $a_n \overset{p}{\longrightarrow} a$ then $a_nX_n \overset{p}{\longrightarrow} aX$
    \item if $X_n \overset{p}{\longrightarrow} X$ and $A_n \overset{p}{\longrightarrow} A$ then $A_nX_n \overset{p}{\longrightarrow} AX$
    \item if $X_n \overset{p}{\longrightarrow} X$, $A_n \overset{p}{\longrightarrow} A$, and $B_n \overset{p}{\longrightarrow} B$ then $A_nX_n + B_n \overset{p}{\longrightarrow} AX + B$
    \item if $X_n \overset{p}{\longrightarrow} X$ and $g$ a continuous function then $g(X_n) \overset{p}{\longrightarrow} g(X)$ \textbf{(continuous mapping theorem)}
\end{itemize}

\subsection{Convergence in $L_p$}
See \url{https://en.wikipedia.org/wiki/Lp_space} for more information (not much covered in class).\\
Convergence in $L_p$ is stronger than convergence in probability. 
\textbf{Counter example} to convergence in probability $\Longrightarrow$ convergence in $L_p$:
\begin{align*}
    \textrm{Let } X_n &= \begin{cases}
        n & \frac{1}{n}\\
        0 & 1 - \frac{1}{n}
    \end{cases}\\
    X_n &\overset{p}{\longrightarrow} 0: \; P(\abs{X_n - 0} \geq \epsilon) = P(X_n = n) = 1/n \longrightarrow 0 \textrm{ as } n \longrightarrow 0\\
    \textrm{but } E(X_n) &= n\frac{1}{n} + 0(1 - \frac{1}{n}) = 1 \Longrightarrow \textrm{ no convergence in } L_p
\end{align*}

\subsection{Convergence in distribution}
A sequence of random vectors, $X_n$, converges in distribution to a random vector, $X_n \overset{d}{\longrightarrow} X$ when
\begin{equation*}
    \lim_{n\longrightarrow \infty} F_{X_n}(x) = F_X(x) \textrm{ at all continuity points in } F_X
\end{equation*}

\begin{itemize}
    \item Convergence in distribution \textbf{does not} imply convergence in probability unless convergence in distribution is to a single point
    \item if $X_n \overset{d}{\longrightarrow} X$ and $g$ a continuous function then $g(X_n) \overset{d}{\longrightarrow} g(X)$ \textbf{(continuous mapping theorem)}
\end{itemize}

\subsubsection{Convergence in probability $\Longrightarrow$ convergence in distribution}
Let $X$ have cdf, $F$, with $t$ a continuity point of F
\begin{align*}
    P(X_n \leq a) \leq& P(X \leq a + \epsilon) + P( \abs{X_n - X} > \epsilon) \textrm{ by lemma}\\
    P(X \leq a - \epsilon) - P(\abs{X_n - X} > \epsilon) \leq& P(X \leq a) \leq P(X \leq a + \epsilon) + P(\abs{X_n - X} > \epsilon)\\
    F_X(a - \epsilon) \leq& \lim_{n\rightarrow \infty} P(X_n \leq a) \leq F_X(a + \epsilon) \textrm{, where } F_X(a) = P(X \leq a)\\
    &\Longrightarrow \lim_{n \rightarrow \infty} P(X \leq a) = P(X \leq a) \Longrightarrow \{X_n \} \overset{d}{\longrightarrow} X
\end{align*}

\subsubsection{Slutsky's theorem}
$A_nX_n + B_n \overset{d}{\longrightarrow} aX + b$ if
\begin{itemize}
    \item $\{X_n\}$ sequence with $X_n \overset{d}{\longrightarrow} X$
    \item $\{A_n\}$ sequence with $A_n \overset{d}{\longrightarrow} A$
    \item $\{B_n\}$ sequence with $B_n \overset{d}{\longrightarrow} b$
\end{itemize}
\subsubsection{Student's t distribution (example use case of Slutsky)} 
$\frac{\sqrt{n}(\bar{X}_n - \mu)}{\hat{\sigma}} \overset{d}{\longrightarrow} N(0, 1)$:
\begin{align*}
    \frac{\sqrt{n}(\bar{X}_n - \mu)}{\hat{\sigma}} &= \frac{\sqrt{n}(\bar{X}_n - \mu)}{\sigma} \frac{\sigma}{\hat{\sigma}}\\
    \textrm{And we know }& \frac{\sqrt{n}(\bar{X}_n - \mu)}{\sigma} \overset{d}{\longrightarrow} N(0, 1)\\
    \textrm{and } & \frac{\sigma}{\hat{\sigma}} \overset{p}{\longrightarrow} 1 \textrm{ since } \hat{\sigma} \overset{p}{\longrightarrow} \sigma\\
    \textrm{So, by Slutsky's theorem } &\frac{\sqrt{n}(\bar{X}_n - \mu)}{\hat{\sigma}} \overset{d}{\longrightarrow} N(0, 1) * 1
\end{align*}
\textbf{This RHS term is referred to as the t-statistic}, which follows a Stuent's t distribution with $n-1$ degrees of freedom. In practice, if the sample is reasonably sized, it won't make a difference using the Normal distribution instead of the Student's t distribution. 

\subsection{Law of large numbers}
For $X_1, X_2, \dots, X_n$ a sequence of i.i.d. random variables with $E(X_i) = \mu$,  $Var(X_i) = \sigma^2$, $\overline{X}_n = \frac{1}{n}\sum_{I = 1}^n X_i$, then for any $\epsilon > 0$
\begin{equation*}
	P(\abs{\overline{X}_n - \mu} > \epsilon) \longrightarrow 0 \textrm{ as } n \rightarrow \infty
\end{equation*}
\textbf{Proof:}
\begin{align*}
	\textrm{First find $\mathbb{E}(\overline{X}_n)$ and $Var(\overline{X}_n)$} \\
	& \mathbb{E}(\overline{X}_n) = \frac{1}{n}\sum_{I = 1}^n \mathbb{E}(X_i) = \mu \\
	& Var(\overline{X}_n) = \frac{1}{n^2}\sum_{I = 1}^nVar(X_i) = \frac{\sigma^2}{n} \textrm{, since $X_i$ independent}
\end{align*}
The desired result now follows immediately from Chebyshev’s inequality, which states
\begin{align*}
	& P(\abs{\overline{X}_n - \mu} > \epsilon) \leq \frac{Var(\overline{X}_n)}{\epsilon^2} = \frac{\sigma^2}{n\epsilon^2} \rightarrow 0 \textrm{ as } n \rightarrow \infty
\end{align*}

\subsection{Central limit theorem}
Most useful form of CLT, which can be used for approximate methods:
\begin{align*}
	\sqrt{n}\frac{(\overline{X}_n - \mu)}{\sigma} &\longrightarrow N(0, 1)\\
	\sqrt{n}(\overline{X}_n - \mu) &\longrightarrow N(0, \sigma^2)
\end{align*}
\textbf{More formal definition and proof:} For $X_1, X_2, \dots, X_n$ a sequence of i.i.d. random variables with $E(X_i) = 0$,  $Var(X_i) = \sigma^2$, and the common cumulative distribution function $F$ and moment-generating function $M$ defined in a neighborhood of zero. Then
\begin{align*}
	\textrm{For } S_n = \sum_{i=1}^nX_i\\
	\lim_{n \rightarrow \infty}P(\frac{S_n}{\sigma \sqrt{n}} \leq x) = \Phi(x)
\end{align*}
\textbf{Proof:} Let $Z_n  = \frac{S_n}{\sigma \sqrt{n}}$. We show the MGF of $Z_n$ tends to the MGF of the standard normal distribution. Since $S_n$ is a sum of independent random variables,
\begin{align*}
	M_{S_n}(t) &= [M(t)]^n \textrm{ and } M_{Z_n}(t) = [M(\frac{t}{\sigma \sqrt{n}})]^n\\
	\textrm{Reminder: } & \textrm{Taylor series expansion of } M(s) = M(0)+sM'(0)+ \frac{1}{2}sM''(0) + \epsilon_s \\
	M(\frac{t}{\sigma\sqrt{n}}) &= 1 + \frac{1}{2}\sigma^2(\frac{t}{\sigma \sqrt{n}})^2 + \epsilon_n \textrm{ with } E(X) = M'(0) = 0, Var(X) = M''(0) = \sigma^2\\
	M_{Z_n}(t) &= (1 + \frac{t^2}{2n} + \epsilon_n)^n\\
	M_{Z_n}(t) &\longrightarrow e^{\frac{t^2}{2}} \textrm{ as } n \longrightarrow \infty \textrm{, by the infinite series convergence to $e^a$}
\end{align*}
Since $e^{\frac{t^2}{2}}$ is the MGF of the standard normal distribution, we have proven the central limit theorem.

\subsection{Delta method}
If $g$ is a differentiable function at $\mu$
\begin{equation*}
    \sqrt{n}(g(\bar{X}_n) - g(\mu)) \overset{d}{\longrightarrow} N(0, g'(\mu)^2\sigma^2)
\end{equation*}
\textbf{Proof}
\begin{align*}
    \textrm{For general } &g \textrm{ and assuming } E(\bar{X}_n) = \mu:\\
    g(\bar{X}_n) \approx& \; g(\mu) + g'(\mu)(\bar{X}_n - \mu) + \frac{1}{2}g''(\mu)(\bar{X}_n - \mu)^2 + \epsilon \textrm{ (Taylor approximation of $g(\mu)$)}\\
    g(\bar{X}_n) - g(\mu) \approx& \; g'(\mu)(\bar{X}_n - \mu) + \epsilon\\
    \sqrt{n}(g(\bar{X}_n) - g(\mu)) \approx& \; g'(\mu)\sqrt{n}(\bar{X}_n - \mu) + \epsilon \textrm{ and we know }\\
    & \sqrt{n}(\bar{X}_n - \mu) \overset{d}{\longrightarrow} N(0, \sigma^2)\\
    & g'(\mu)\sqrt{n}(\bar{X}_n - \mu) \overset{d}{\longrightarrow} N(0, g'(\mu)^2\sigma(2))\\
    \textrm{So } \sqrt{n}(g(\bar{X}_n) - g(\mu)) &\overset{d}{\longrightarrow} N(0, g'(\mu)^2\sigma(2))
\end{align*}
\textbf{Note:} if we find that $g'(\mu) = 0$, then repeat this process with the second derivative, $g''(\mu)$. The end result is a formula that converges in distribution to a scaling of a random variable, $Z^2$ which follows a $\chi_1^2$ distribution.


\section{Estimation}
The following section provides an overview of methods for estimating population parameters, $\theta$, using functions of the data ("estimators"), $T(X_1, \dots, X_n)$

\subsection{Mean Squared Error}
The \textbf{Mean Squared Error (MSE)} can be used to evaluate our estimators.
\begin{align*}
    MSE(T, \theta) &= E_\theta[(T - g(\theta))^2]\\
    &= E_\theta(T^2) - 2g(\theta)E_\theta(T) + g(\theta)^2\\
    &= Var_\theta(T) + E_\theta(T)^2 + 2g(\theta)E_\theta(T) + g(\theta)^2\\
    &= Var_\theta(T) + (E_\theta(T) - g(\theta))^2\\
    &= Var_\theta(T) + Bias^2_\theta(T) \textrm{, where } Bias_\theta(T)= E_\theta(T) - g(\theta)
\end{align*}

\subsection{Method of Moments estimator}

To generate a method of moments estimator
\begin{itemize}
    \item Calculate a moment using the moment generating function of the assumed distribution. Any moment, $k$, can be used, but lower moments will typically lead to an estimator distribution with lower variance
    \begin{equation*}
        E(X^k) = g(\theta)
    \end{equation*}
    \item Invert this expression to create an expression for the parameter(s) in terms of the moment
    \begin{equation*}
        g^{-1}(E(X^k)) = \theta \Longrightarrow f(E(X^k)) = \theta \textrm{, where } f(x) = g^{-1}(x)
    \end{equation*}
    \item Insert the sample moment into this expression, thus obtaining estimates of the parameters in terms of data
    \begin{equation*}
        \hat{\theta} = f(\frac{1}{n}\sum X_i^k) \; \;\textrm{, by LNN } \frac{1}{n}\sum X_i^k \overset{p}{\longrightarrow} E(X^k)
    \end{equation*}
    \item Use the delta method to determine what the method of moments estimator converges to in distribution
    \begin{equation*}
        \sqrt{n}(f(\frac{1}{n}\sum X_i^k) - \theta) \overset{d}{\longrightarrow}  N(0, f'(E(X_i^k))^2Var(X_i^k)^2)
    \end{equation*}
\end{itemize}
Methods of moment estimators are not uniquely determined, nor must they exist. The motivation for subsequent estimators is to help us pick the estimator with the smallest possible variance.

\subsection{Maximum likelihood estimator}
The \textbf{maximum likelihood estimator} constructs an estimator, $\hat{\theta}_{MLE}$, that maximizes the likelihood function with respect to $\theta$. \\ \\
The \textbf{likelihood function}, $L(\theta)$ is the joint density or probability mass function, $f(X, \theta)$ evaluated at the data, $\{X_i, \dots, X_n\}$. Assuming the data is $i.i.d.$:
\begin{equation*}
    L(\theta) = \prod_{i=1}^nf(X_i, \theta)
\end{equation*}
The typical method to generate a maximum likelihood estimator
\begin{itemize}
    \item Construct the likelihood function
    \begin{equation*}
        L(\theta) = \prod_{i=1}^nf(X_i, \theta)
    \end{equation*}
    \item Take the log of the likelihood function (usually leading to a function that is easier to derive)
    \begin{equation*}
        log(L(\theta)) = l(\theta) = \sum_{i=1}^nlog(f(X_i, \theta))
    \end{equation*}
    \item Take the derivative of the log-likelihood function with respect to $\theta$
    \begin{equation*}
        \frac{d}{d\theta}l(\theta) = \sum_{i=1}^n\frac{d}{d\theta}log(f(X_i, \theta))
    \end{equation*}
    \item Find critical points of this function and determine that one is a maximum
    \begin{align*}
        0 =& \sum_{i=1}^n\frac{d}{d\theta}log(f(X_i, \hat{\theta}))\\
        0 =& \sum_{i=1}^n\frac{d^2}{d\theta^2}log(f(X_i, \hat{\theta})) \textrm{, checking if } \hat{\theta} < 0
    \end{align*}
\end{itemize}
See next section on \textbf{Fischer Information} for guidance on the asymptotic distribution of the maximum likelihood estimator

\subsection{Fisher Information}
The \textbf{information} that data, $X$, contains about parameter, $\theta$ is defined by
\begin{equation*}
    I(\theta) = E_\theta \left [ \left (\frac{d}{d\theta}log(f(X, \theta)) \right )^2 \right ]
\end{equation*}
\begin{itemize}
    \item Fisher Information assumes \textbf{differentability} and \textbf{existence of the second moment}
    \item $\frac{d}{d\theta}log(f(X, \theta))$ is called the \textbf{score} function
\end{itemize}
\subsubsection{Properties of Fischer Information}
\begin{align*}
    (i) \; & E_\theta \left [ \left (\frac{d}{d\theta}log(f(X, \theta)) \right ) \right ] = 0:\\
    &  E_\theta \left [\left (\frac{d}{d\theta}log(f(X, \theta)) \right ) \right ] = 
    \int \frac{d}{d\theta}log(f(x, \theta))f(x, \theta)dx = \int \frac{f'(x, \theta)}{f(x, \theta)} f(x, \theta)dx =
    \int f'(x, \theta) dx\\
    & E_\theta \left [\left (\frac{d}{d\theta}log(f(X, \theta)) \right ) \right ] = \frac{d}{d\theta} \int f(x, \theta)dx =  \frac{d}{d\theta}*1 = 0\\ \\
    (ii) \; & I(\theta) = Var \left ( \frac{d}{d\theta}log(f(X, \theta)) \right ):\\
    & Var \left ( \frac{d}{d\theta}log(f(X, \theta)) \right ) = E_\theta \left [ \left (\frac{d}{d\theta}log(f(X, \theta)) \right )^2 \right ] - E_\theta \left [ \left (\frac{d}{d\theta}log(f(X, \theta)) \right ) \right ]^2\\
    & Var \left ( \frac{d}{d\theta}log(f(X, \theta)) \right ) = I(\theta) - 0^2 = I(\theta)\\ \\
    (iii) \; & I(\theta) = - E_\theta \left [\frac{d^2}{d\theta^2}log(f(X, \theta)) \right ]\\
    & \frac{d}{d\theta}log(f(x, \theta)) = \frac{f'(x, \theta)}{f(x, \theta)} \Longrightarrow \frac{d^2}{d\theta^2}log(f(x, \theta)) = \frac{f(x, \theta)f''(x, \theta) - f'(x, \theta)^2}{f(x, \theta)^2}\\
    & E\left [ \frac{d^2}{d\theta^2}log(f(x, \theta)) \right] = \int \frac{f(x, \theta)f''(x, \theta) - f'(x, \theta)^2}{f(x, \theta)^2}f(x, \theta)dx = \int f''(x, \theta) - I(\theta)\\
    & E\left [ \frac{d^2}{d\theta^2}log(f(x, \theta)) \right] = -I(\theta) \textrm{, since } \int \frac{d^2}{d\theta^2} f(x, \theta) = \frac{d^2}{d\theta^2} * 1 = 0\\ \\
    (iv) \; & I_{X, Y}(\theta) = I_X(\theta) + I_Y(\theta) \textrm{ for $X, Y$ independent}\\
    & \textrm{\textbf{Corrolary:} } I_n(\theta) = nI_1(\theta) \textrm{ for } X_1, \dots, X_n \; i.i.d \textrm{ with } I_{1}(\theta) \textrm{ the Information based on one data}\\ 
    & \textrm{\textbf{Note:} Information increases with larger sample!}\\ \\
    (v) \; & \textrm{\textbf{Cramer-Rau-Fisher Inequality:} }Var(T(X)) \geq \frac{g'(\theta)^2}{I(\theta)} \textrm{ for } E(T(X)) = g(\theta)\\
    & Cov[T(X), \frac{d}{d\theta}log(f(X, \theta))] = E[T(X)\frac{d}{d\theta}log(f(X, \theta))] \textrm{, using property 1}\\
    & Cov[T(X), \frac{d}{d\theta}log(f(X, \theta))] = \int T(x)f'(x, \theta)dx = \frac{d}{d\theta} \int T(x)  f(x, \theta)dx = \frac{d}{d\theta}E(T(X)) = \frac{d}{d\theta}g(\theta) = g'(\theta)\\
    &g'(\theta)^2 \leq Var(T(X))Var \left (\frac{d}{d\theta}log(f(X, \theta)) \right) = Var(T(X))I(\theta) \textrm{ by correlation inequality: } \rho^2 \leq 1\\
    &Var(T(X)) \geq \frac{g'(\theta)^2}{I(\theta)}
\end{align*}
\subsubsection{The "Big" theorem: Asymptotic distribution using Fischer Information}


\subsection{Bayes estimator}
\subsection{Key theorems}
\subsection{Consistency}
\subsection{Efficiency}
\subsection{Sufficiency}

\section{Hypothesis testing}
\subsection{Likelihood ratio}
\subsection{Neyman-Pearson lemma}
\subsection{Uniformly Most Powerful tests}
\subsection{Confidence intervals}

\section{Analysis of categorical data}
\subsection{Chi-Square Test}
\subsection{Fisher's Exact Test}







\end{document}
